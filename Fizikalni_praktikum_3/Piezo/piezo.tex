\documentclass[12pt,a4paper]{article}
\usepackage[slovene]{babel}
\usepackage{geometry}
\geometry{legalpaper, margin=1in}
\usepackage{graphicx}
\usepackage{soul}
\usepackage{xcolor}
\usepackage{gensymb}
\usepackage{alltt}


\begin{document}

\title{Piezoelektri\v cnost}
\author{Urh Trinko}
\maketitle

\newpage


\section{UVOD}

Piezoelektričnim kristalom se ob mehanski obremenitvi spremeni električna ploarizacija, velja pa tudi obratno, saj zunanje električno polje povzroči deformacijo kristala. Piezoelektrični kristali se uporabljajo v napravah za meritev spremembe tlaka/sil, mikrofone, generatorje ultrazvoka ter pri delovanju tunelskega mikroskopa.
\\
\\
Pri vaji bomo opazovali piezoelektri\v cni odziv ploščice iz piezokeramike, ko nanjo delujemo z neko silo. Ko na ploščisco pritisnemo s silo F je ustrvarjena (talčna) napetost T enaka $\frac{F}{S}$, zato nastane polarizacija $P_{3} = d \cdot T$. Ob tem se med gostoto električnega polja D in polarizacijo $P_3$nvspostavi zveza:

$$D = \epsilon \epsilon_0 E + P_3$$

($\epsilon$ - dielektrična konstanta piezoelektrika)
\\
\\
Za naboj na posamezni plošči keramike velja iz ena\v cbe $q = D S$ sledi:

$$q = \textcolor{red}{\frac{\epsilon \epsilon_0 S}{b} U} + d \cdot F$$
\\
Rdeči člen predstavlja le drug način za zapis naboja na ploščatem kondenzatorju:

\begin{equation}
	{C = \frac{\epsilon \epsilon_0 S}{b}}
\end{equation}

(S - ploščina kondenzatorja, b - debelina, C - kapaciteta)
\\
Iz tega sledi, da lahko enačbo za naboj $q$ preoblikujemo v:

\begin{equation}
	{q = C U + d \cdot F}
\end{equation}
\\
Za časovno spreminjanje napetosti pa velja:

\begin{equation}
	{U_{s}(t) = s U_{0} e^{-t/\tau}}
\end{equation}

($s = +$ pri obremenitvi, $s = -$ pri razbremenitvi, časovna konstanta $\tau$ je enaka $RC$, kjer je $R$ upor, $C$ pa kapaciteta kondenzatorja)

\section{POTREBŠČINE}

\begin{itemize}
	\item merilna valjasta posoda s piezoelektrično keramiko
	\item elektrometrskiojačevalnik z baterijskim napajalnikom
	\item digitalni osciloskop
	\item USB kjuč
	\item uteži za 200g, 500g in 1kg 
\end{itemize}

\section{NALOGA}

\begin{enumerate}
	\item Izmeri dielektrično konstanto piezoelektrične keramike.
	\item Izračunaj piezoelektrični koeficient keramike.
\end{enumerate}























\end{document}